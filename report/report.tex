% !TeX root = report.tex
\documentclass[a4paper,12pt]{report}

\usepackage{tabularx}
\usepackage{alltt, fancyvrb, url}
\usepackage{graphicx}
\usepackage[utf8]{inputenc}
\usepackage{float}
\usepackage{hyperref}
\usepackage{caption}
\usepackage{listings}
\usepackage{xcolor}


% Questo commentalo se vuoi scrivere in inglese.
\usepackage[italian]{babel}

\usepackage[italian]{cleveref}

% Centra verticalmente i contenuti delle righe
\renewcommand\tabularxcolumn[1]{m{#1}}


\title{Analisi e progettazione di un gestionale WEB per gestire articoli tramite RFID}


\author{
\\Matteini Mattia
\\Olaiya Kelvin
\\Paganelli Alberto
}
\date{\today}

\begin{document}

\maketitle

\tableofcontents


\chapter{Introduzione}

\section{Analisi generale}


\section{Progettazione}
\subsection{Raffinamento diagramma delle classi}


\chapter{Base di dati}

\section{Definizione delle specifiche in linguaggio naturale ed estrazione dei concetti principali}

\section{Progettazione Concettuale}

\section{Progettazione Logica}


\chapter{Software}

\section{Analisi}

\subsection{Diagramma delle classi}


\chapter{Guida utente}



\end{document}